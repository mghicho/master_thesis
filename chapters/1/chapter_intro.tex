\startfirstchapter{Introduction}
\label{chapter:introduction}

The word \textit{IoT}, which stands for Internet of Things, describes an environment where every "\textit{thing}" is connected to the Internet. The use of the word "\textit{thing}" suggests that no device is exempt. The number of opportunities that a world where every device, small or large is connected to a global network brings, is beyond anyone's imagination. As always, along with opportunities come challenges that need to be resolved for such a scenario to function properly. Almost all layers of communication protocol need to be redesigned to handle these challenges, together with an effort to come up with applications that can use this extra-connectivity in its full potential.

The "\textit{things}" that IoT refers to can be everyday objects. From home appliances to cars and roads. But perhaps the importance of IoT can never be understood without considering sensors. Sensor networks used in monitoring and metering applications are perhaps one of the main and first applications of IoT. Sensors that can be attached to anything and can be anywhere with different requirements. For some emergency response related applications, latency is the key factor, whereas for some long term monitoring applications, power consumption is more important. But one thing most IoT application have in common is their large density of nodes. For this reason, any IoT solution has to be able to function in densely populated networks. 

The deployment and maintenance cost of wired networks makes it almost inevitable to think that this super connected world can be only achieved using wireless communication. Now, one can grasp an idea of how dominant the IoT can be by just looking at future standards in different wireless communication domains. In cellular communications, Device to Device (D2D) communication is obviously a major part of the next generation of LTE standard known as 5G \cite{doppler2009device}. There has also been dedicated effort to support the dense network of nodes that an IoT environment brings and to support the very low latency that some time critical IoT applications like autonomous cars require. Just like cellular communications, wirelessLAN community have also taken big steps toward future IoT networks. IEEE 802.11ah standard has been designed with IoT in mind. It not only introduces new sub 1GHz bands for sensor networks, also addresses the intense channel contention that a dense network brings. For the first time among .11 standards, it can support up to 6000 nodes connected to one access point \cite{khorov2015survey}. 

Another promising IoT technology is Physical Layer Network Coding (PNC). It can present a robust solution for multi-hop communications. Multi-hop communication is essential to IoT applications that have large number of sensors scattered in large geographical areas such as sensors monitoring highways, railroads, rivers and skyscrapers. The nature of these applications does not allow a direct connection to a sink node for every sensor. Thus, packets have to be relayed to the sink node with the help of other nodes. 


\section{Research Objectives and Contributions}
This thesis is trying to solve two IoT problems. For single hop transmissions, the hidden terminal problem in infrastructure-based IoT scenarios under IEEE 802.11ah is considered. And for multi-hop transmissions, the implementation of a physical layer network coding system for relay transmissions is the focus.

\subsection{RSS-Based Grouping Strategy}
The IEEE 802.11ah standard proposed for dense sensor networks, uses a sub 1GHz band that can result in transmission ranges as high as 1 km. In a network with such a large radius and density, the traditional hidden terminal problem becomes more serious and conventional solutions do not seem to address the problem in its full extent. Recognizing this challenge, an RSS-based grouping strategy is proposed to categorize nodes in a way that the hidden terminal problem can no longer be an issue. The performance gain of the proposed method is then evaluated by an analytical model and simulations.


\section{Practical Physical Layer Network Coding Implementation}
Despite being proposed a decade ago, physical layer network coding have not yet been fully used in practical applications. One of the reasons behind this delay is implementation challenges that a real-life PNC system has to face. Implementation challenges of PNC are addressed and studied in this dissertation. Providing a solution for these challenges, feasibility of a real-life PNC system is proven by a software defined radio implementation on GNURadio platform. Different performance parameters are then measured in both simulation and experiment. 


\section{Dissertation Organization}

This thesis is organized as following:
\begin{description}
\item[\textbf{Chapter \ref{chapter:introduction}}] gives an introduction about the concept of IoT, and explains breifly the problems that this thesis is trying to solve.
\item[\textbf{Chapter \ref{chapter:single}}] explains the IEEE 802.11ah standard and the proposed RSS-based grouping strategy for GS-DCF protocol.
\item[\textbf{Chapter \ref{chapter:multi}}] explains the Physical Layer Network Coding technology and it's practical challenges and presents a Software Defined Radio implementation of a PNC system.
\item[\textbf{Chapter \ref{chapter:eval}}] presents the evaluation and results for both works presented in chapter \ref{chapter:single} and \ref{chapter:multi}. It includes both simulation and experimental results.
\item[\textbf{Chapter \ref{chapter:concl}}] is the conclusion. It also presents areas for further research and improvement of this work.
\end{description}


\section{Bibliographic Notes}
The work in chapter \ref{chapter:single} was published in \cite{ghasemiahmadi2017rss}. 