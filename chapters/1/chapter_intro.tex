\startfirstchapter{Introduction}
\label{chapter:introduction}

The word \textit{IoT}, which stands for Internet of Things, describes an environment where every "\textit{thing}" is connected to the Internet. The use of the word "\textit{thing}" suggests that no device is exempt. The number of opportunities that a world where every device, small or large is connected to a global network brings, is beyond anyone's imagination. As always, along with opportunities come challenges that need to be resolved for such a scenario to function properly. Almost all layers of communication protocol need to be redesigned to handle these challenges, together with an effort to come up with applications that can use the extra-connectivity in its full potential.
%\pagebreak
\section{My Claims}
Something must be new in this work, no matter how small, since you are getting a graduate degree for it! Tell me about it clearly and succintly right now, just as you did in the abstract. Make an impact here. How about something like the following box:

I make \textit{four} claims which
my dissertation validates:
\\

\framebox{%
\parbox{5in}{
	My new algorithm to solve the problem of doing nothing include these important new features whose practical applicability can be proved both formally and empirically:
	\begin{enumerate}
	\item first feature;
	\item second feature;
	\item everything is much easier to understand, and therefore, easier to implement correctly.
	\end{enumerate}
}}
\\

\noindent Claim 1 and claim 2 are \textit{quantitative} - they will be proven by experiment.

\noindent Claim 3 is \textit{qualitative} - they will be demonstrated by argument.

\subsection{The Importance of My Claims}

Some very important positive consequences
arise from the validation of the above claims.
It is these consequences that comprise a significant
positive contribution to research in the field
of whatever the field is.
\\

\noindent Claim 1 implies that:
\begin{enumerate}
\item{Something profound which applies to:
	\begin{itemize}
	\item {something excellent;}
	\item {something important.}
	\end{itemize}}
\item{Something else just as profound.}
\end{enumerate}

\noindent Claim 2 implies that:
\begin{itemize}
\item{Repeat as above if necessary and useful.}
\end{itemize}

\noindent The consequence of claim 3 is that:
\begin{itemize}
\item{There must be something good coming out of all this work!}
\end{itemize}

\section{Agenda}

This section provides a map of the dissertation
to show the reader where and how it validates
the claims previously made. Here is where I am also presenting my own style of organization which may be totally different from what your supervisor thinks. However, trust me, this is a good solid beginning for a structure. Your supervisor may ask you to change it, but will still appreciate what you have! For each of the chapters below I also give a short summary of what the main focus should be and then I expand on it  a bit within the chapter itself.

\begin{description}
\item[\textbf{Chapter 1}] give and introduction about the concept of IoT
\item[\textbf{Chapter 2}] describes in details the open problem which is to be tackled together with its context
\item[\textbf{Chapter 3}] explaines the rss-based grouping strategy
\item[\textbf{Chapter 4}] is going to explain the pnc implementation
\item[\textbf{Chapter 5}] is the pnc practical experiments with photos of usrps
\item[\textbf{Chapter 6}] is the evaluation resutls of both works.
\item[\textbf{Chapter 7}] is the conclution
\end{description}

The list above is not complete. Chapter 3 actually includes a lot more, as I could not resist placing in it a few \LaTeX examples to help you along. This document is not a primer for \LaTeX, but there is no harm done in giving a little help.
