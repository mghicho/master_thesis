\startchapter{Conclusions}
\label{concl}

talks conclusions stuff. here is the conclusion of the grouping manuscript:
The hidden terminal problem under GS-DCF has been addressed in this paper. An RSS-based grouping scheme has been proposed. The proposed scheme will categorize users in RSS-based groups so nodes in the same group can be close enough that the probability of having a hidden terminal is very low.
The performance of RSS-based grouping has been then studied and compared to random and optimal grouping schemes using analytical and simulation results presenting different metrics.

Another problem in future infrastructure networks that can be solved with RSS-based groups is small message size of most sensing data exchanges, which will reduce channel efficiency. \cite{bianchi2000performance} showed that small packet size will reduce the overall throughput. RSS-based grouping schemes can easily utilize clustering algorithms where each group is a cluster and a cluster head in each group can act as a relay for other nodes. We intend to cover this application in our future work.