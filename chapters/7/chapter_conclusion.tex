\startchapter{Conclusions}
\label{chapter:concl}

Two efficient wireless networking solutions supporting IoT have been the main focus for this thesis. In single-hop communications, super dense networks and IEEE 802.11ah as a promising standard for sensor networks are studied and the hidden terminal problem under GS-DCF has been addressed. An RSS-based grouping scheme has been proposed. The proposed scheme will categorize users in RSS-based groups so nodes in the same group can be close enough that the probability of having a hidden terminal is very low.
The performance of RSS-based grouping has been then studied and compared to random and optimal grouping schemes using analytical and simulation results presenting different metrics.

Another problem in future infrastructure networks that can be solved with RSS-based groups is small message size of most sensing data exchanges, which will reduce channel efficiency. \cite{bianchi2000performance} showed that small packet size will reduce the overall throughput. RSS-based grouping schemes can easily utilize clustering algorithms where each group is a cluster and a cluster head in each group can act as a relay for other nodes. We intend to cover this application in our future work.

The second half of the thesis has focused on multi-hop transmissions and recognized physical layer network coding as a robust technology for many sensor network solutions. Since practical implementation has been missed in many PNC designs, this thesis proposed solutions to deal with the challenges that a real-life PNC system has to face and validates its feasibility using a testbed implementation on USRP devices. Experiment results show that MPNC can outperform the traditional relay with a throughput gain ranging from 75\% to 80\% for the two and five-hop cases, and around 150\% for the three- and four-hop cases in the real testbed.

There are still many unsolved challenges and improvement areas around MPNC in general and its practical implementation. Since the relay only receives a combined version of its surrounding nodes, unconventional channel coding algorithms are no longer useful. Design of new channel coding algorithms for the multiple access phase of PNC can improve its performance by a large margin. Multi-path channel and fast fading are also practical issues not addressed by this thesis. Following the same idea proposed in this thesis but modeling the channel as an FIR filter can overcome these challenges. Integrating MPNC with MIMO, accounting for random-hop distances, and uni-directional traffic are other areas that this work can be further improved.