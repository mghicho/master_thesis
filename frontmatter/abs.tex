\newpage
\TOCadd{Abstract}

\noindent \textbf{Supervisory Committee}
\tpbreak
\panel

\begin{center}
\textbf{ABSTRACT}
\end{center}

The promise of Internet of Things (IoT) and mass connectivity has brought many applications and along with them many new challenges to be solved. Recognizing sensor networks as one of the main applications of IoT, this dissertation focuses on solutions for IoT challenges in both single-hop and multi-hop communications. In single-hop communications, the new IEEE 802.11ah and its Group Synchronized Distribution Coordination Function (GS-DCF) is studied. GS-DCF categorized nodes in multiple groups to solve the channel contention issue of dense networks. An RSS-Based grouping strategy is proposed for the hidden terminal problem that can arise in infrastructure-based single hop communications. For multi-hop communications, Physical Layer Network Coding (PNC) is studied as a robust solution for multi-hop packet exchange in linear networks. Focusing on practical and implementation issues of PNC systems, different challenges have been addressed and a Software Defined Radio (SDR) PNC system based on USRP devices is proposed and implemented. Finally, extensive simulation and experimental results are presented to evaluate the performance of the proposed algorithms in comparison with currently used methods.